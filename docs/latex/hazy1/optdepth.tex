\chapter{OPTICAL DEPTHS AND RADIATIVE TRANSFER}
% !TEX root = hazy1.tex

\section{Overview}

Line transfer is relatively unimportant in
low-density clouds such as \hii\ regions
and planetary nebulae.
Radiative transfer can be important in other
environments,
such as nova shells and the broad-line region of active nuclei,
where excited states of hydrogen have significant
populations and subordinate
lines become optically thick.
In other cases grains are present and all
lines can be absorbed by background opacity.
All radiative transfer effects
are included in the treatment of line formation,
including line thermalization,
destruction by background opacities, pumping by the incident continuum,
and escape from the cloud.

It is necessary to iterate upon the solution if emission lines are
optically thick since total optical depths are not known on the first
iteration.
The default is for a single iteration.
This is
often adequate for low-density nebulae such as planetary nebulae or \hii\ regions.
A second iteration is sometimes enough to establish a fairly
accurate line optical depth scale for many resonance transitions.
Further
iterations are usually needed when subordinate lines are also optically
thick.
The \cdCommand{iterate to convergence} command
will iterate until the optical-depth scale is well defined.

Line-radiation pressure cannot be computed accurately until the total
line optical depths are known,
so this quantity is only meaningful after
the first iteration.
\Cloudy\ will stop if the internal radiation pressure
exceeds half of the surface gas pressure in a constant-pressure
model since
such a geometry is unstable unless it is self-gravitating.
The radiation
pressure is not allowed to exceed half the gas pressure on the initial
iterations of a multi-iteration constant-pressure model.
This is to prevent
the calculation from stopping when the optical depth scale is
not yet well converged.

The following sections outline various commands that affect the
radiative transfer and discuss several of the model atoms that are available.

\section{Atom overview}

The \cdCommand{atom} commands change details of the physical treatment of several
model atoms that are incorporated in the code.

\section{Atom FeII [options]}
\label{sec:AtomFeIICommand}

This command adjusts details of the model Fe$^+$ ion.
The default is to
compute the lowest 16 levels, which produce IR emission, as in the real
ion, but to do higher levels, which produce optical and UV emission, using
the simplified and very fast scheme outlined by \citet{Wills1985}
AGN3 describes \feii\ emission in Section 14.5.

When any of the \cdCommand{atom FeII} commands are entered
the Wills et al. scheme is turned off.
A complete model, using 371 levels and described by \citet{Verner1999}, is used instead.
This is far more accurate but also \emph{much} slower.
\citet{Baldwin2004} show some examples of \feii\ spectra of AGN.

\emph{N.B.} the keyword is \cdCommand{FeII}, not \cdCommand{Fe2},
to avoid scanning the number 2 off the command line.
Both \cdCommand{FeII} and \cdCommand{Fe II} (with a space) are accepted.

\subsection{Atom FeII levels}

This changes the number of levels used by the model atom.
The upper
limit is 371 levels and is the default when the large \feii\ model is used.
When the large atom is not used the default is to compute
the lowest 16
levels and treat the remainder of the levels with the \citet{Wills1985}
simple model.
Decreasing the number of levels will speed up the execution
time [roughly proportional to $n^2 \log(n)]$ at the expense of a degraded
simulation of the physics.

\subsection{Atom FeII trace}

This turns on debugging printout for each call to the model atom.

\subsection{Atom FeII redistribution [options]}

The keyword \cdCommand{redistribution} will change
the form of the redistribution
function for various lines within the model atom.

A keyword to specify which set of lines to change is required.
This
is either \cdCommand{resonance} (any line decaying to the ground term)
or \cdCommand{subordinate}
(transitions between excited levels).
The type of redistribution function
to use is also specified.
The options are \cdCommand{PRD} (partial redistribution),
\cdCommand{CRD} (complete redistribution with Doppler core only),
and \cdCommand{CRDW} (complete
redistribution with damping wings).
(The underscore indicates a space.)

The keywords \cdCommand{redistribution show} tells the code to print
the current default redistribution functions.

\subsection{Atom FeII simulate}

This will cause results from the atom to be simulated.  The large model
atom is not actually called.  This is very fast and is a debugging aid.

\subsection{Atom FeII slow}

The keyword \cdCommand{slow} will cause the atom to always be reevaluated.  Normally
the code only reevaluates the atom when the local conditions have changed
significantly.

\subsection{Atom FeII output options}

\feii\ emission is an exercise in uncontrolled complexity.  Hundreds of
thousands of lines contribute to what often appears as a pseudo continuum.
It is not practical to study the list of \feii\ lines
except in a few especially
simple situations such as \hii\ regions.

Three options are available for understanding output from the \feii\ atom.
Most often \feii\ emission is seen as a blended continuum rather than
individual lines.  The general idea is to try to reduce the flood of
information to a manageable level by distilling the emission into what an
observer would actually see.

\emph{\feii\ bands in the main output.}  A series of \feii\ ``bands'' are
automatically entered into the main emission-line output.
Each band represents the total \feii\ emission integrated
over all lines that lie within a range of wavelengths.
This emission appears
in the output with the label ``Fe2b'' and a wavelength close to the center
of the band.
The bands are chosen to represent features that an observer
might be able to measure.

The list of bands is contained in the file
\cdFilename{FeII\_bands.ini}
which is included in the data directory.
This file is intended to be easily changed
by the user.
It explains the format of the information that is needed.
There is no limit to the number of \feii\ bands that can be specified.

\emph{The \cdCommand{save FeII} family of commands}
are described on page \pageref{sec:CommandSaveFeII}
and provide a way to save some predictions.
The following gives an overview of some of these commands.

\cdCommand{save FeII lines}  This reports the intensities
of all emission
lines predicted by the model atom.
The resulting file is very large and
mainly useful for debugging the model atom or understanding where within
the atom a particular feature originates.

\cdCommand{save FeII continuum}  This reports the
total \feii\ emission as a continuous spectrum.
Here a range of wavelengths is broken into a number
of intervals and the total \feii\ emission within each interval is added
together.
The result represents what would be observed by a spectrometer
with a particular resolution and was used to create the plots
in \citet{Baldwin2004}.
Some details, such as the wavelength range and resolution,
are changed with the \cdCommand{set FeII continuum} command
described on page \pageref{sec:CommandSetFeIIContinuum}.

\cdCommand{atom FeII continuum from 1000\AA\ to 7000\AA\  in 1000 cells}
This adjusts the lower and upper bounds and the resolution of the
\feii\ continuum
saved with the \cdCommand{save FeII continuum} command.
The numbers are the lower
and upper limits to the wavelength range in Angstroms and the number of
wavelength cell that occur over this range.

\section{Atom H2 options}

The large model of the \htwo\ molecule, described in \citet{Shaw2005},
will be included if any of the family of \cdCommand{atom H2}
commands appears.
The much
faster three-level model outlined by \citet{Tielens1985a},
\citet{Burton1990}, \citet{Draine1996},
and expanded
by \citet{Elwert2006}, is used by default.
The large molecule is
represented by several thousand levels producing roughly half a million
lines.
AGN3 describes some properties of \htwo\ in section 8.3
and appendix A6.

\emph{NB}  The number ``2'' appears in the keyword for this command.
Any
numerical parameters that appear on the \cdCommand{atom H2}
commands must appear after
this two---the code will check that the first number
parsed off the command line is the number~2.

\subsection{The \cdCommand{set H2} command}

Some details of the physical treatment of the \htwo\ molecule can be
changed with the \cdCommand{set H2} command.

\subsection{\htwo\ output options}

Strong \htwo\ lines will appear in the main emission-line output.
The \cdCommand{save H2} command has many other output options.
The
section in Part 2 of this document describing calling the code as a
subroutine describes several routines that will return predictions of the
large molecule.
The \cdCommand{set line precision} command
will add more significant figures to the wavelengths printed in the main
printout.
This may help isolate a particular line.

\subsection{Atom H2}

With no options, the only effect is to include the large model of the
\htwo\ molecule.

\subsection{Atom H2 chemistry [simple; full]}

This changes how the interactions between the \htwo\ molecule
and the rest of the chemical network are treated.
By default, or if the keyword
\cdCommand{full}
appears, then the fully self-consistent formation and destruction rates
are used when the large \htwo\ molecule is enabled.
If the keyword
\cdCommand{simple} occurs
then expressions from \citet{Tielens1985a} are used instead.

\subsection{atom H2 collisions [options]}

These commands change various collisional processes within
the \htwo\ molecule.

\cdCommand{atom H2 ortho para collisions on/off}

This turns off
ortho-para changing
collisions with gas particles.

\cdCommand{atom H2 orH2 collisions options}

\cdCommand{atom H2 paH2 collisions options}

These commands determine which of the \htwo\ -- \htwo\ collision data
sets is used.
The default is the \citet{LeeH2H22008} set, which can also
be chosen with the \cdCommand{ORNL} option.
The keyword \cdCommand{Le Bourlet} selects the
\citet{LeBourlot1999} data set.
The keyword \cdCommand{ohH2} adjusts the ortho data and the
keyword \cdCommand{paH2} adjusts the para data set.
One of these keywords must be specified.  Both cannot be adjusted
with the same command.

\cdCommand{atom H2 collisional dissociation on/off}
turns on or off collisional dissociation.
The default is to include it using the estimates given in
\citet{Shaw2005}.
These rates are all only guesses and represent an
uncertainty.

\cdCommand{atom H2 grain collisions on/off} turns off downward
transitions induced by collisions with grains.

By default the code will uses guesses of collisional rate coefficients
using the g-bar method.
Collisional deexcitation for the g-bar transitions
are turned off with the \cdCommand{atom H2 gbar} command.

\subsection{Atom H2 gbar [ off; on]}

The g-bar approximation is a rough relationship between the energy of
a line and its collision rate coefficient.  This can be used to guess a
collisional deexcitation rate coefficient when no real data exist.
This command turns this guess off or on.  It is on by default.

\subsection{Atom H2 levels }

This changes the number of electronic levels within the \htwo\ molecule.
The default is seven and includes the ground and first six bound singlet
electronic states.
This is also the largest number of levels.  At minimum
is three levels, which is sufficient to include the Lyman and Werner bands
in the UV.
This are a necessary minimum number of levels to include the
correct photodissociation processes.
If no number appears but the keyword
\cdCommand{large} does then the code will use the upper limit.

\subsection{Atom H2 limit -4  }

Calculating the level populations and line spectrum of the large \htwo\ molecule is computationally expensive.
The code tries to save time by not
computing the populations when the abundance of \htwo\ is negligible.  This
command changes the limit for the smallest \htwo/H$_{\mathrm{tot}}$ ratio.
The full model
will be computed when the ratio is greater than this limit.
The number
is interpreted as the linear ratio if it is greater than zero and the log
of the ratio if it is less than or equal to zero.
The keyword
\cdCommand{off} turns
off the limit so that the large model of the molecule is always evaluated.
The default limit is 10$^{-8}$, small enough for the large molecule to be
computed across the entire \citet{Tielens1985a} standard PDR
model that is part of the code's test suite.

When the \htwo\ abundance is below this limit the photodissociation,
heating, and cooling rates, are evaluated using expressions in
\citet{Tielens1985a}.
\htwo\ level populations are set to their LTE value so that
self-shielding in the electronic bands is still computed.

\subsection{Atom H2 matrix 43}

Populations of the lower ro-vibration states of the ground electronic
level are determined by two schemes.
The first, and most straightforward,
is the solution of a complete set of balance equations by solving
the system
of balance equations with a master equation approach.
The time needed to solve the linear
algebra increases as a power of the number of levels
so we need to keep
the number of levels as small as possible.
High levels are treated by back
substitution, starting from the highest level with X and proceeding
downwards.
This command sets the total number of levels that are computed
within the matrix.
Levels higher than this will be treated with back
substitution.

If the keyword \cdCommand{off} (note the leading space) or
\cdCommand{none} appears, or if the
number of levels is less than 1, the matrix will not be used.
It the keyword
\cdCommand{all} appears then all levels within X will be done this way.

\subsection{Atom H2 noise [mean, standard deviation, seed]}

This multiplies the rates for collisional processes within the \htwo\ molecule
by a Gaussian random number so that $r' = r\;10^{{\mathrm{rand}}} $.
Here $r$
is the correct rate coefficient and \cdTerm{rand} is an Gaussian
distributed random number.  The first two optional numbers on the command
line set the mean and standard deviation for the Gaussian random numbers.
The first optional number is the mean, with a default of 0.  The second
optional number is the standard deviation with a default of 0.5.  The last
optional number is the seed for the random number generator, which must
be an integer greater than 0.
If the seed is not specified then the system
time is used to generate a random seed.

\subsection{Atom H2 thermal [simple; full]}

This changes how the heating and cooling by the \htwo\ molecule
are treated.
By default, or if the keyword \cdCommand{full} appears,
then the fully self-consistent
heating and cooling rates are used when the large \htwo\ molecule
is enabled.
If the keyword \cdCommand{simple} occurs then expressions
from \citet{Tielens1985a} are used instead.

\subsection{Atom H2 trace [options]}

This turns on trace information concerning the \htwo\ molecule.
The optional
keywords \cdCommand{full}, \cdCommand{iterations},
and \cdCommand{final} will give full information, an overview
of iterations during the convergence, and only final results respectively.
If the keyword \cdCommand{matrix} occurs the code will print
the contents of the matrices
that are used for solution of lower levels within the ground electronic
state.

\subsection{atom H2 output options}

Roughly half a million lines are predicted when the large
\htwo\ molecule is included.
The main emission-line printout includes all significant lines
produced in the ground electronic state but does not include electronic
transitions.
The \cdCommand{print line H2 electronic} command
will include these lines.
The family of \cdCommand{save H2} commands
provides ways to save information such as column densities,
the emission-line spectrum, and details of the effects of \htwo\ on the
conditions in the cloud.

\section{Atom H-like [options]}

These are used to change some details in the treatment of atoms of the
H-like isoelectronic sequences.\footnote{This was the \cdCommand{hydrogenic}
command in versions 90 and before.}
Atoms
of the He-like sequence have two bound electrons and include He$^0$,
Li$^+$, through Zn$^{+28}$.
Atoms of the H-like sequence
have one bound electron and include \h0, He$^{+}$, Li$^{+2}$,
through Zn$^{+29}$.

This implementation of the H-like sequence was initially part of Jason
Ferguson's PhD thesis and is described in \citet{FergusonFerland1997},
\citet{Ferguson2001}, and \citet{BottorffBaldwin2002}.
The physics was expanded
to resolve $nl$ terms by Ryan Porter during a visit to IoA
Cambridge in Fall 2008.
Any number of levels up to $n$ of 400 can be computed.

The He-like sequence was developed by Ryan Porter as part
of his thesis and it is described in \citet{Bauman2005},
\citet{Porter2005}, and \citet{PorterFerland2007}.
Ryan Porter unified the two sequences
and expanded the H-like sequence to include all of the physics included
in the He-like sequence.

The two sequences are now unified so the same
\cdCommand{atom xx-like} commands work
for both.
One of the iso-sequences must appear on the command line.  The
keywords are \cdCommand{H-like} and \cdCommand{He-like}.
Only one iso-sequence can be modified
with a single command.
If the name of an element does not also appear then
the entire iso-sequence will be changed.
If the name does appear then only
that atom is modified.  The following are some examples
\begin{verbatim}
c only change hydrogen itself
atom H-like hydrogen levels 8
c change the entire He-like sequence
atom He-like levels 9
\end{verbatim}
The model atoms include a certain number of resolved and collapsed levels.
The lower $n_{resolved}$ levels are $nl$ resolved.
Another $n_{collapsed}$ levels,
which replace the $nl$ terms with an $n$ configuration,
are above the resolved levels.
The physics of the resolved levels is exact while that of the
collapsed levels is more approximate since the entire \emph{n} configuration is
treated as a single level.
By default only lines produced by resolved levels
are included in the printout.

\subsection{Atom H-like collisions\dots }

Collisional processes, both between levels ionization, are turned off
with this command.  They are all on by default.  This command is mainly
used for debugging.  Separate collisional processes can be turned off with
the following options.  Only one option is recognized per command line so
multiple commands are needed to turn off several processes.  If no
sub-options are recognized then all collisional processes are disabled.
This command turns off collisions for all elements along the H-like
isoelectronic sequence.

\cdCommand{atom H-like collisions l-mixing off}\footnote{This was the 2s2p option in versions 94 and before.  The 2s2p option
still exists for backward compatibility.}
This turns off \emph{l}-mixing $2s-2p$ collisions.

\cdCommand{atom H-like collisional ionization off}
This command turns off collisional
ionization by thermal electrons of all levels,
except for the very highest level.
Collisional ionization from the highest level is not turned off
to allow the atom to have some coupling to the continuum.
Ionization by
cosmic rays is not affected by this command.

\cdCommand{atom H-like collisional excitation off}
This command turns off collisional
excitation of all levels, except for 2s-2p.

\cdCommand{atom H-like collisions off}
All three collisional processes will be turned
off if none of the three keywords are recognized.

\emph{Warning!}  The code will require a very number of zones
if collisions
are turned off in an optically thick cloud with a large
$(n \gg 15)$ hydrogen atom.
Collisions will normally hold populations of very highly excited
levels to values very near LTE.
The FIR and radio lines will have very
small line optical depths due to the correction for stimulated emission.
When collisions are absent, the normal tendency of departure coefficients
to increase with principal quantum number means that FIR and radio lines
will strongly mase.
The code dynamically adjusts the zoning to prevent
these maser optical depths from diverging to minus infinity.
A very large
number of zones will be required to spatially resolve the masing region.
This is a totally artificial, not physical, effect.
The solution is to
not turn off collisions with a large atom when performing a
simulation with more than a trivial thickness.

\subsection{Atom H-like continuum lowering off  }
This command disables continuum lowering processes due to particle packing,
Debye shielding, or Stark broadening.  The processes are active by default
and implemented following \citet{Bautista00}.  This command applies
to the entire isoelectronic sequence.

\subsection{Atom H-like damping off  }

Rayleigh scattering, continuum scattering due to the extreme damping
wings of Lyman lines, can be turned off with the \cdCommand{damping off} option.
Rayleigh scattering is a significant opacity source in clouds that have
large column densities of neutral material
($N$(\h0 $) > 10^{23} \mathrm{cm}^{-2}$).

\subsection{Atom H-like levels 15 [element iron]}

This sets the number of levels.
The atom can extend up to any principle
quantum number between the limits $4 < n \le 400$.
The size is limited mainly
by the available memory and computer time.
The default highest quantum level is $n = 15$.
Increasing the
number of levels allows a better representation of the collision physics
that occurs within higher levels of the atom at the expense of longer
execution times and greater memory requirements.

If no number appears on the \cdCommand{atom H-like levels} command,
but the keyword
\cdCommand{large} or \cdCommand{small} does,
then either 50 or 10 levels will be used.
If the
keyword \cdCommand{very small} also appears then
the smallest possible atom,
up to $n = 4$ to include H$\beta$, is computed.
These keywords provide a version-independent
method of insuring that the code uses the largest or smallest possible
number of levels.

By default this command changes all elements along the hydrogenic
isoelectronic sequence.
If the keyword \cdCommand{element} appears together with the
name of an element, only the model atom for that particular element will
be changed.
For example, the following would set the full isoelectronic
sequence to a small number of levels, then reset hydrogen, helium,
and iron to a large number.
\begin{verbatim}
atom H-like levels small
atom H-like levels large element hydrogen
atom H-like levels large element helium
atom H-like levels large element iron
\end{verbatim}
The number of levels can only be set once at the very start of a
calculation when space is allocated for the hydrogenic arrays.
If the code
is used to run a grid of models then only the first occurrence of
\cdCommand{atom H-like levels} is honored and all following occurrences
are ignored.

\emph{Warning!}  Note that the command
\begin{verbatim}
atom H-like levels large
\end{verbatim}
will set all \LIMELM\ hydrogenic atoms to a large number of levels.
This would
require roughly half a gigabyte of memory for the H-like sequence along,
and would be very slow on today's computers.
It is best to set only the
most important elements to large levels.

\subsection{Atom H-like Lyman lines}

This set of commands changes the treatment of the \hi\ Lyman lines.

\cdCommand{atom H-like Lyman pumping off}  turns off
continuum radiative pumping of the Lyman lines of \hi.
This is meant to take into account the possible
presence of Lyman lines in absorption in the incident continuum.
This
command only changes \hi\ and not the entire H-like sequence.

\cdCommand{atom H-like Lyman pumping scale 3}
The continuum radiative pumping rate
of the Lyman lines of \hi\ will be multiplied by the scale factor
that appears on the line.
This is meant to take into account the possible presence of
Lyman lines in emission or absorption in the incident continuum.
The pumping
rate is normally set by the intensity of the coarse continuum at the line
wavelength.
This number that appears on the command line is a scale factor
that multiplies this continuum.\footnote{In versions C08 and before a value of 0 would produce black Lyman lines.
Starting in C10 values $\le 0$ are interpreted as logs
so a value of 0 will result in a scale factor of unity.}
Numbers $\le 0$ are interpreted as a log and
the \cdCommand{log} keyword will force interpretation as a log.
A value of greater than 1 would correspond to totally Lyman lines
in emission.
A scale factor of
2 would simulate the presence of stellar Lyman emission lines with peak
intensities twice those of the neighboring continuum.
This command only
changes \hi\ and not the entire H-like sequence.

\cdCommand{atom H-like extra Lyman 1000}  Atoms and ions of the
H-like and He-like
isoelectronic sequences use complete multi-level model atoms.
The number
of levels included is limited mainly by processor speed and
available memory.
Higher Lyman lines (used here to mean permitted lines
that connect directly
to ground) have little impact on emission since they
scatter and are degraded
into Balmer lines and L$\alpha $.
However, an absorption spectrum will show them
as a series of lines converging onto the photoionization continuum from
the ground state.
The code includes a large number of ``extra'' Lyman lines,
included as absorbers with optical depths output with
the \cdCommand{save line optical depth} command,
but not treated as part of the multi-level atoms.
The default number of higher Lyman lines is 100, and
this can be changed to any number with this command.

\subsection{Atom H-like matrix [LowT, populations]}

This command has been deprecated and now has no effect.  The low-temperature solver has been removed,
and matrix inversion is now used in all cases with non-trivial density or ionization.

\subsection{Atom H-like redistribution [options]}

The keyword \cdCommand{redistribution} will change the form of the redistribution
function for various lines within the model atom.
This command is only
used if you are not happy with the default redistribution functions.

A keyword to specify which set of lines to change is required.
This is one of \cdCommand{alpha} (the $2p - 1s$ transition), \cdCommand{resonance} (any higher Lyman line
decaying to the ground term), or \cdCommand{subordinate}
(all Balmer, Paschen, etc lines).
The type of redistribution function to use must also be specified.
The options are \cdCommand{PRD} (partial redistribution), \cdCommand{CRD} (complete redistribution
with Doppler core only), and \cdCommand{CRDW}
(complete redistribution with damping wings).
(The underscore indicates a space.)

The keyword \cdCommand{show} tells the code to print the current default
redistribution functions.

There is at present a fundamental uncertainty in the computation of the
line radiation pressure for transitions such as L$\alpha $.
For a simple two-level
atom with incomplete redistribution, it has long been known that the
line-width is proportional to ($a\tau)^{1/3}$
(\citealp{Adams1972}, \citealp{Harrington1973}; $a$ is
the damping constant).  It is also easily shown that for complete
redistribution and a frequency independent source function that the line
width would be determined by inverting the Voigt function, and hence
proportional to ($a\tau)^{1/2}$.
Line interlocking, whereby scattered Balmer line
radiation broadens the upper level of L$\alpha$
(\citealp{Hubbard1985}), can
alter the line width, as can collisional effects when the density is high
enough for distant collisions to broaden the line.
These effects cause
major differences in radiation pressure and emergent flux (factors of
several) for \la, which can easily have an optical depth of
$10^7$--$10^9$, when
Balmer lines are also optically thick.  T
his command determines which
approximation is used.
The default condition is incomplete redistribution,
which minimizes the line width and radiation pressure.  This issue is
discussed further in \citet{Elitzur1986}.

\subsection{Atom H-like TopOff 6 [add scale]}

This toppoff off or on.
The default is to be on and the keyword \cdCommand{off} disables topoff.
Topoff is necessary to obtain the correct total
radiative recombination rate coefficient with a finite number of levels.
Because only a finite number of levels can be computed the sum
of the total
recombination coefficient will be less than the sum to infinity.
This difference must be added somewhere to conserve the
total recombination rate.

\section{Atom He-like [options]}

The following commands change treatments of some of the physical details
of how the models of atoms along the He-like isoelectronic sequence (the
species He$^0,\dots$ C$^{+4}$, N$^{+5}$, etc.).
Details are given in \citet{Bauman2005},
\citet{Porter2005}, and \citet{PorterFerlandMacAdam2007}.

\subsection{Atom He-like collapsed levels x}

This sets the number of collapsed levels, the high-$n$ levels that are
assumed to be well $l$ mixed and hence can be treated without full
consideration of $L$ and $S$ states.  This is faster than solving for the
resolved states, but the $l$-mixed approximation is only correct at higher
densities.  One is the minimum number of collapsed levels, as well as the
default.

\subsection{Atom He-like collisions \dots.}

Collisional processes between levels and collisional ionization are turned
off with this command.
Separate collisional processes can be turned off
with the following options.
Only one option is recognized per command line
so multiple commands are needed to turn off several processes.
If no
sub-options are recognized then all collisional processes are disabled.
This command turns off collisions for all elements along the He-like
isoelectronic sequence.

\cdCommand{atom He-like collisions l-mixing off}  This command
turns off all
collisions within the same $n$ level for all elements in the helium-like
isoelectronic sequence.
Both $L$- and $S$-changing collisions are turned off.

\cdCommand{atom He-like collisions l-mixing Pengelly [Vrinceanu]}  This
changes to the \citet{PengellySeaton1964} formalism rather than
currently the default,
based on \citet{Vrinceanu2001} for $L > 2$.

\cdCommand{atom He-like collisions l-mixing thermal [ averaging, no averaging]}
The \cdCommand{no thermal averaging} option calculates the
\citet{Vrinceanu2001}
collision strengths at an impact energy equal to $kT$,
rather than averaging
over a Maxwellian distribution.
This is the default when a small atom is
used.
Thermally averaged collision strengths are used when the atom is
large.

\cdCommand{atom He-like collisions excitation}  This command
turns off collisional
excitation, for all $n_u \not= n_l$ transitions,
for all elements in the helium-like
isoelectronic sequence.

\subsection{Atom He-like continuum lowering off  }
This command disables continuum lowering processes due to particle packing,
Debye shielding, or Stark broadening.  The processes are active by default
and implemented following Bautista \& Kallman (2000).  This command applies
to the entire isoelectronic sequence.

\subsection{Atom He-like dielectronic recombination [off] }

This option turns dielectronic recombination off for the entire
isoelectronic sequence.
Dielectronic recombination is included by default.
This command has no effect without the \cdCommand{off} option.

\subsection{Atom He-like error generation X }

Monte Carlo error analysis can be performed with this command.
Randomly
generated Gaussian errors are applied to atomic data.
The parameter X is
an integer used to set the seed of the random number generator.

The pessimistic option will choose the largest of two standard deviations
for each piece of atomic data.
The full command would look something like
\begin{verbatim}
atom he-like error generation pessimistic 5
\end{verbatim}
where the number is the seed for the random number generator. Without the
pessimistic keyword the default optimistic values are used.

\subsection{Atom He-like FSM}

For $L > 2$, levels with the same $n$ and $L$ but different $S$
are strongly
mixed due to the spin-other-orbit interaction.
This option modifies
transition probabilities to account for this fine-structure mixing.
The
effect on multiplet emissions is negligible.
See \citet{Bauman2005} for
a discussion and calculation of this effect in a zero-density $J$-resolved
helium atom.

\subsection{Atom He-like gbar options}

The code employs various forms of the $\bar g$
approximation to fill in collision strengths for those transitions with
no quantal calculations.
This command changes which approximation is used.
The options are \cdCommand{Vriens} for the \citet{Vriens1980} and
\cdCommand{off} to set this to zero.

\subsection{Atom He-like levels 4 [element iron]}

This sets the number of $n$ levels.
The argument is the principal quantum
number $n$ for the highest level.
It must be three or greater.

If no number appears on the command, but the keyword \cdCommand{large} or
\cdCommand{small} does,
then the number of levels will be large enough to extend
to either $n = 40$ or 3.
These keywords provide a version-independent method of insuring that
the code uses a large or small number of levels.
Actually, the atoms are
coded so that there is no limit to the number of levels that
can be included, other than the memory and computer time requirements.

Another keyword, \cdCommand{huge}, is used solely for collisionless Case~B
calculations, such as those in \citet{Bauman2005}.
It causes \Cloudy\ to
bypass some system memory allocation for a faster, less memory-intensive,
calculation of radiative-cascade spectra.
The code cannot be used for other
purposes since it disables everything not directly related to the radiative
cascade of helium.
Disabling this physics is the only direct effect of
this command.
It is still necessary to set the desired number of levels.

The model atom resolves $n$-levels into $nLS$ levels,
and the 2 $^3P$ term is split into three $nLSJ$ levels.
For a given maximum principal quantum number
$n_{\max}$, there will be a total of $n_{\max}^2 + n_{\max} +1 nLS$ levels.  The default $n_{\max}$ for He$^0$ is 6, resulting in
43 $nLS$ levels, $n_{\max} = 5$ for 31 $nLS$ levels
for C, N, O, Fe, and Zn, and $n_{\max} = 3$ for 13 $nLS$ levels
for all remaining elements.
For reference, to include all $nLS$ levels within the $n = 2, 3,
4, 5$, and 6 levels, the atom will need 7, 13, 21, 31, and 43 levels,
respectively.
Increasing $n_{\max}$ allows a better representation of the atom's
emission,
especially the collision physics that occurs within higher levels
of the atom,
but at the expense of longer execution times and greater memory
requirements.

By default this command changes the number of levels for all elements
along the He-like isoelectronic sequence.  If the keyword \cdCommand{element} appears
together with the name of an element, only the model atom for that particular
element will be changed.  For example, the following would set the full
isoelectronic sequence to a small number of levels, then reset helium and
iron to a large number.
\begin{verbatim}
atom He-like levels small
atom He-like levels large element helium
atom He-like levels large element iron
\end{verbatim}

The number of levels can only be set once at the very start of a grid
of calculations.
This is because space is allocated for the needed arrays
only one time per core load.
If the code is used to run a series of models
then only the \cdCommand{atom He-like levels} that occur in the first simulation will
be honored and all following occurrences will be ignored.

The model atom does not give a good representation of lines that come
from the highest $n$ level, which is always a ``collapsed'' level.
Only lines
coming from the first $n - 1$ levels will be printed at the end of the
calculation.

\emph{Warning!}  Note that the command
\begin{verbatim}
atom He-like levels large
\end{verbatim}
will set all 29 He-like atoms to a large number of levels.
This would
require roughly half a gigabyte of memory and would be very slow
on today's computers.
It is best to set only the most important elements to large
levels.
Depending on the application, this may be only He itself (when
$\sim 10^4$ K gas is considered) or the more abundant second
and third row elements (for hot gas and X-ray applications).

There are also collapsed levels---levels that do not resolve
$S$ or $L$ but
assume that the states within $n$ are well-mixed.
These collapsed levels
bring together all of the individual $nLS$ terms as one pseudo-level.
The
recombination coefficient into this pseudo-level is the sum of recombination
coefficients into the individual terms plus the recombination topoff.
Transition probabilities from this pseudo-level are calculated as follows.
\begin{equation}
A(n_u \to n_lL_lS)= \frac{\sum_{L_u=L_pm 1} g_{L_{u}S} A(n_uL_uS\to
n_lL_lS)}{\sum_{L_u=L_l\pm 1}}.%49
\end{equation}
This causes the collapsed level to behave exactly as if it were a set of
resolved terms populated according to statistical weight.

\subsection{Atom He-like Lyman 1000}

Atoms and ions of the H-like and He-like isoelectronic sequences use
complete multi-level model atoms.
The number of levels included is limited
mainly by processor speed and available memory.
Higher Lyman lines (used
here to mean permitted lines that connect directly to ground) have little
impact on the emission,
since they scatter and are degraded in Balmer lines
and L$\alpha $.
However, an absorption spectrum will show them as a series of lines
converging onto the continuum from the ground state.
The code includes
a large number of ``extra'' Lyman lines, included as absorbers with optical
depths output with the \cdCommand{save line optical depth} command,
but not treated as part of the multi-level atoms.
The default
number of higher Lyman lines is 100, and this can be changed to any number
with this command.

\subsection{Atom He-like matrix [LowT, populations]}

This command has been deprecated and now has no effect.  The low-temperature solver has been removed,
and matrix inversion is now used in all cases with non-trivial density or ionization.

\subsection{Atom He-like no recombination interpolation}

The code normally derives recombination coefficients by interpolating
on a table that lives in the data directory.
This command tells the code
to compute recombination coefficients on the fly by integrating over the
photoionization cross section.

\subsection{Atom He-like redistribution [options]}

The keyword \cdCommand{redistribution} will change the form
of the redistribution
function for various lines within the model atom.

A keyword to specify which set of lines to change is required.
This
is one of \cdCommand{alpha} (the $2\, ^1P - 1\, ^1S$ transition),
\cdCommand{resonance} (any higher Lyman
line decaying to the ground term),
or \cdCommand{subordinate} (all Balmer, Paschen,
etc lines).
The type of redistribution function to use must also be
specified.
The options are \cdCommand{PRD} (partial redistribution),
\cdCommand{CRD} (complete
redistribution with Doppler core only),
and \cdCommand{CRDW} (complete redistribution
with damping wings).
(The underscore indicates a space.)

The keyword \cdCommand{show} tells the code to print the current default
redistribution functions.

\section{Atom [h-like \OR{} he-like] [options]}

This is a nominally a repeat of the above sections, but is directly
based on the source structure in October 2009.  Should be merged with
the above, but may be useful as a local to-do list when checking for
consistency.


\subsection{Atom [h-like \OR{} he-like] collisions}

Option to turn collisions off, all are on by default.  
Command can accept only one option at a time.
\begin{description}
\item[EXCI] turn off collisional excitation
\item[IONI] turn off collisional excitation
\item[2S2P|2P2S]  deprecated form of \cdCommand{ATOM ISO COLLISIONS 
L-MIXING} (2P2S only applies for the H-like iso-sequence)
\item[L-MIxing] change from 2s2p to l-mixing (if H-like iso-sequence).
Otherwise \cdCommand{THERmal} option uses thermal averaging from
Vrinceau et al, or \cdCommand{PENG} or \cdCommand{OFF} required.
\end{description}


\subsection{Atom [h-like \OR{} he-like] DAMP}

Turn off absorption due to Ly alpha damping wings for H-like
iso-sequence.

\subsection{Atom [h-like \OR{} he-like] DIEL}

Sets which set of data to use for dielectronic
recombination (currently only options are \cdCommand{OFF} or nothing).

\subsection{Atom [h-like \OR{} he-like] LEVE}

Number of levels for iso-sequence.  When keyword ELEMENT
appears, scan off element name and change levels only for that one.
when there is no ELEMENT then set all in iso to same number.  Argument
is number or a keyword, either LARGe (or limit) or SMALl (or COMPact).
COLLapsed option.

\subsection{Atom [h-like \OR{} he-like] ERROr GENEration}

Rates will be modified by a randomly generated
error that falls within the range specifically set for each rate (or
set of rates).  Has PESSimistic option.

\subsection{Atom [h-like \OR{} he-like] FSM}

Turn on fine structure mixing of spontaneous decays for
He-like isosequence (see Bauman et al 2003).

\subsection{Atom [h-like \OR{} he-like] GBAR}

Set cross section of higher levels, options are VRIEns, NEW, or OFF.

\subsection{Atom [h-like \OR{} he-like] LYMAN PUMP}

Possible values are OFF or SCALE, to set multiplicative factor for all
continuum pumping of H I Lyman lines, account for possible emission in
the line -- only affects H I not entire H-like iso sequence.

\subsection{Atom [h-like \OR{} he-like] NO RECOmbination INTErp}

Generate recombination coefficients on the fly, rather than by table
interpolation.

\subsection{Atom [h-like \OR{} he-like] REDI}

There are three options, PRD, CRD, and CRDW, representing partial
redistribution, complete redistribution (doppler core only, no wings)
and complete with wings, respectively.  Additional options are ALPH
(to set value for Lyman $\alpha$), RESO (to set value for other
resonance lines), and SUBO (to set value for subordinate lines).  SHOW
to prints diagnostic output to confirm selected options.

\subsection{Atom [h-like \OR{} he-like] MATR}

This command has been deprecated.  It now has no effect.

\subsection{Atom [h-like \OR{} he-like] TOPO} 

Turns topoff on or (with keyword OFF) off.

\section{Case A [options]}

This has the same options as the Case~B command but sets the \la\ optical
depth to a very small value by default.
This does not turn off induced
processes, which are normally ignored when Case A is assumed.
You would
use the \cdCommand{no induced processes} command to do that.

\section{Case~B [tau ly alpha = 9; options]}

This command is used to check the line emission from the hydrogen and
helium atoms in the Case~B limit (AGN3 section 4.2).
This command \emph{should
not} be used in any model that is supposed to represent a real physical
environment.
It is intended only to provide an easy way to check predictions
of the code against simple, more limited, calculations.
In particular,
when this is used the Lyman-line optical depths will be made artificially
large.
This may affect the ionization and temperature of the gas.

With no options this command sets the inner optical depth of
\la\ for all
atoms and ions of the H- and He-like iso-electronic sequences to
$10^5$ so
that even a one-zone model will be close to
Case~B.\footnote{Before version
96 the default optical depth was $10^9$.  This caused
extreme L$\alpha $ behavior in a grain-free \hii\ region.  The lower value is a better
estimate of the physics that occurs in an actual \hii\ region.}
The optional number
is the log of the L$\alpha $ optical depth.
One-sided escape probabilities are
used so the total escape probability is simply that for the inward direction.
In keeping with the Case~B approximation the \cdCommand{Case~B} command suppresses
excited-state line optical depths.

The atoms include all collisional processes.
Case~B does not define
the population of the ground or first excited state so a true comparison
with Case~B results should have collisions from these levels turned off.
This is done with the Hummer and Storey option (with the key \cdCommand{humm}), to allow
comparison with their 1987 and 1995 papers.
Collisions from the ground
and first excited states \emph{are} included if this second option is not specified.
Collisions between levels with $n\ge  3$ \emph{are} included unless the
\cdCommand{atom H-like collision off}
or \cdCommand{atom He-like collision off} commands are given.  Collisions
between the $2s$ and $2p$ levels are always included unless the
\cdCommand{atom H-like collisions 2s2p off} command is given.

In the case of the He-like isoelectronic sequence
the \cdCommand{Case~B} command
sets the optical depths in the singlet Lyman lines to a large value.
The
Hummer \& Storey option has no effect on the He-like sequence.

The \cdCommand{no Pdest} option turns off destruction of
Lyman lines by background opacity.

There are several side effects of this command that may after the spectrum
or physical conditions in unexpected ways.
The large L$\alpha $ optical depth will
often result in an especially strong radiation field within this line.
This affects the gas through photoionization of excited metastable states
of H and He and of those elements
with a small enough ionization potential.
The \cdCommand{no photoionization} option on the \cdCommand{Case~B}
command tells the code not to
include photoionization from excited metastable states like the 2s level
of \h0.
But these strong diffuse fields will also strongly affect the level
of ionization of the gas, making the resulting ionization equilibrium a
fiction.
Optically thin gas is actually described by Case C
(\citealp{Ferland1999}, \citealp{LuridianaEtAl09} and
AGN3 Section 11.4) where continuum pumping enhances Balmer lines.
The large
Lyman-line optical depths that result from the \cdCommand{Case~B}
command will prevent
continuum resonant pumping of the atom.  Beware.

This command is used in the test suite to perform classical PDR
simulations which ignore the \hplus\ region and assume that the PDR is
illuminated by a radiation field with no hydrogen-ionizing radiation.
Classical PDR calculations do not include a full treatment of H or He so
their ions are only produced by cosmic rays.
\Cloudy\ does a full treatment
of the physics of \h0\ and He$^0$
and will find a thin layer of highly ionized
gas.
This is produced by Lyman-line pumping into excited states, especially
metastable 2s, which are then photoionized by relatively low-energy light.
The process stops when the Lyman line optical depths become large enough
for the atom to become self-shielding.
The \cdCommand{Case~B} command will stop this
process from ever becoming important by starting with large Lyman line
optical depths.
(The process would not happen in a realistic calculation
which included the \hplus\ region since Lyman line optical depths
through the \hplus\ region are large.)

This command is included in the ``homework problem'' PDR simulations
in the test suite to
stop Lyman line pumping from occurring.
A better solution would be to include
the PDR as a additional layer outside the \hplus\ region.
The Lyman line
optical depths across the ionized gas will be large and prevent continuum
pumping from occurring in the PDR.
Combining the \hii\ region and PDR also
results in a self-consistent calculation
(\citealp{Abel2005} and \citealp{Abel2008} ).

\section{Case C [options]}

This has the same options as the Case~B command but sets the
L$\alpha$ optical depth to a very small value by default.
Case C is described by \citet{Ferland1999} and
\citet{LuridianaEtAl09}.

\section{Diffuse fields [outward, OTS]}

This specifies which method is to be used to transfer the diffuse fields,
the emission from gas within the computed structure.
The options are \cdCommand{outward only} and \cdCommand{OTS}.

The \cdCommand{OTS} option takes into account optical depths
in both the inward and outward directions.
The \cdCommand{OTS} option has a \cdCommand{SIMPLE} option which will
do a very simple OTS approximation without taking optical depths into
account.
All diffuse fields with energies capable of ionizing hydrogen
are assumed to do so, and those with smaller energies freely escape.
This is intended as a debugging tool.

If \cdCommand{outward} is chosen then the code will check for a number.
This determines which of the many forms of the outward-only
approximation (\citealp{Tarter1967}) is used.
The default\footnote{OTS was the default in version 86 and before.} is 2.  This is intended for testing the code.

This choice does not strongly affect the predicted emission-line spectrum
but it does change the temperature at the illuminated face of the cloud.

\section{Double optical depths}

On second and later iterations the code uses the total optical depths
of the computed structure to find the outwardly-directed radiation field.
This command doubles the total optical depth so that the shielded face of
the cloud becomes the mid-plane of a structure that is twice as thick as
the computed cloud.

This original purpose of this command was to simulate a geometry in which
ionizing radiation strikes the plane-parallel cloud from both sides.
Examples are a L$\alpha$ forest cloud or the diffuse ISM.
The total line and
continuum optical depths are set to twice the computed optical depth at
the end of the iteration.
The computed model is then one half of the cloud
and the other half of the cloud is assumed to be a mirror image
of the first half.
Doubling the total line and continuum optical depths at the end of
the iteration is the \emph{only} effect of this command.
Physical quantities such
as the physical thickness, column densities, or line emission
\emph{are not} affected.

This approximation makes sense if the cloud is optically thick in lines
but optically thin (or nearly so) in continua.
Lines such as the L$\alpha $
transitions of He I and He II can be important sources
of ionizing radiation.
Their transport will be handled correctly in this limit when this command
is used.
Continuum transport out of the cloud will also be treated
correctly, but attenuation of the incident continuum will
\emph{not} be if the
cloud is optically thick in the continuum.

The second use of this command is when the outer edge of a computed
structure is not the other edge of the cloud.
A typical PDR calculation
is an example.
The calculation starts at the illuminated face and continues
until the gas becomes cool and molecular.  The stopping point often does
not correspond to the outer boundary of the molecular cloud, but rather
is a point that is ``deep enough'' for a given study.  The optical depths
are always computed self-consistently.  On second and later iterations the
total optical depths are normally those of the computed structure.  Near
the shielded face the outward optical depths will be small and radiation
will freely escape in the outward direction.  The gas temperature may fall
dramatically due to the enhanced cooling resulting from the free escape
of line photons.
In real PDRs considerable neutral or molecular material
probably extends beyond the stopping point so that line photons do not freely escape.
The shielding effects of this unmodeled extra material can be
included with this command.
Then, the shielded face of the cloud will
correspond to the mid-plane of the overall structure and lines will not
artificially radiate freely into the outer (unmodeled) hemisphere.

\section{Iterate [2 times]}

This specifies the number of iterations to be performed.
The default
is a single iteration, a single pass through the model.
At least a second
iteration should be performed in order to establish the correct total optical
depth scale when line transfer or radiation pressure is important.
Two
iterations are sometimes sufficient and will be done if no numbers are
entered on the command line.
A comment will be printed after the last
iteration if the total optical depth scale has not converged and further
iterations are needed.

\subsection{Number of iterations}

There is a slight inconsistency in how the code counts the number of
iterations.
The way it functions in practice is what makes the most sense
to me.

The word \emph{iterate} is from Latin for ``again.''
So the true number of
``agains'' should be one less than the total number of calculations of the
cloud structure.
When the \cdCommand{iterate} command is not entered there is one
calculation of the structure and so formally no iterations.
 If any one
of the following commands is entered:
\begin{verbatim}
iterate
iterate 0
iterate 1
iterate 2
\end{verbatim}
then exactly two calculations of the structure will be done.
If the number
on the line is two or greater, then the number will be the total number
of calculations of the structure.

\subsection{Iterate to convergence [max =7, error =.05] }

This is a special form of the \cdCommand{iterate} command
in which the code will
continue to iterate until the optical depths have converged or a limit to
the number of iterations has been reached.
The optional first number on
the line is the maximum number of iterations to perform with a default of
10.  The second optional number is the convergence criterion.
The default
is for relative optical depths to have changed by less than \autocv\ between
the last two iterations.
The optional numbers may be omitted from right
to left.
If all transitions are optically thin then only a second iteration
is performed.

\subsection{Convergence problems  }

The code generally will not converge if it has not done so within ten
or so iterations.
The most common reason for convergence problems is that
the specified column density or thickness causes the simulation to end
very near a prominent ionization front.
In this case very small changes in the
physical conditions result in large changes in the optical depths.  This
is a physical, not numerical, problem.
The code will not have convergence
problems if an optical depth is used as a stopping criterion instead.

\section{No scattering opacity}
\label{sec:CommandNoScatteringOpacity}

This turns off several pure scattering opacities.  These include
scattering by grains, electron scattering, and the extreme damping wings
of Lyman lines (Rayleigh scattering).
When scattering opacity is included
and an open geometry is computed the scattering opacity is assumed to
attenuate the incident radiation field as
$\left( {1 + 0.5\,\tau _{scat} } \right)^{ - 1} $
rather than $\exp \left( { - \tau } \right)$ (\citealp{Schuster1905}).

Scattering can be neglected in a spherical geometry with gas fully
covering the source of ionizing radiation.
Scattered photons are not really
lost but continue to diffuse out with (perhaps) a slight shift in energy.
Electron scattering is generally the most important scattering opacity in
a grain-free mixture.
If $ \tau _{scat}  \le 1$
then it is reasonable to consider electron scattering as a heating and
cooling process but not as an absorption mechanism if the energy shifts
are not large (i.e., $ h\nu \ll mc^2$) and the geometry is spherical
(this is not correct for $\gamma$-ray energies,
of course).
\Cloudy\ is not now designed to work in environments that are
quite Compton thick, but should work well for clouds where the electron
scattering optical depths are less than or of order unity.

When this command
is entered scattering processes such as Compton energy exchange 
and grain scattering are still included
as heating, cooling, and ionization processes, but not as extinction sources.
(Thermal and ionization effects of Compton scattering are turned off with
the \cdCommand{no Compton} command).
The \cdCommand{no scattering opacity} command is automatically
generated when \cdCommand{sphere} is specified.

\section{Turbulence = 100 km/s [log, dissipate]}

This enters a microturbulent velocity $u_{turb}$.
The velocity is given in
km s$^{-1}$ on the command line although the code works with
cm s$^{-1}$ internally. The turbulent line width $u_{turb}$ is zero by default,
but the value that is entered must be larger than zero.
If the optional keyword \cdCommand{log} appears then the number
is interpreted as the log of the turbulence. Alternatively, you can
also enter the keyword \cdCommand{equipartition} which is discussed
further below.

Turbulent pressure is included in the equation of state when the total
pressure is computed.\footnote{Turbulence was not included as a pressure term in versions 06.02
and before.}
The \cdCommand{no pressure}
option on this command says not to include turbulent pressure in the total
pressure.

Turbulence affects the shielding and pumping of lines.
Fluorescent
excitation of lines becomes increasingly important for larger turbulent
line widths since a larger part of the continuum can be absorbed by a line.
Line pumping is included as a general excitation mechanism for all lines
using the formalism outlined by \citet{Ferland1992} and described further in
a section of Part 3.
The line-center optical depth varies inversely with
the line width velocity $u$ so the effects of line optical depths
and trapping become smaller with increasing line width.
Larger $u$ inhibits self-shielding.

\subsection{Definitions }

The Doppler width of any line that is broadened by both thermal motions
and turbulence that is given by a Gaussian is given by the quadratic sums
of the thermal and turbulent parts,
\begin{equation}
u = \sqrt {u_{th}^2  + u_{turb}^2 }  = b = \sqrt 2 \sigma\;
 [\cmps ].
\end{equation}
Here $b$ is the Doppler parameter used in much of the UV absorption-line
literature, $\sigma $ is the standard deviation
(often called the dispersion) for
a normal distribution, and $\sigma^2$ is the variance.
For a thermal distribution
of motions the average velocity along the line of sight
(\citealp{Mihalas1978}, equation 9-35, page 250)
and the most probable speed (\citealp{Novotny1973}, p 122)
of a particle with mass $m$ are both given~by
\begin{equation}
u_{th}  = \sqrt {2kT/m}\;
[\cmps ].
\end{equation}
This command sets $u_{turb}$.
Note that with these definitions the full-width half-maximum
(FWHM) of a line is equal to
\begin{equation}
u\left( {FWHM} \right) = u\sqrt {4\ln 2}\;
[\cmps ].
\end{equation}

\subsection{Turbulent pressure}

For an ideal gas the thermal pressure is $P = nkT$
while the energy density is $1/2 nkT$
per degree of freedom, so for a monatomic gas is $U = 3/2 nkT$.
Turbulent motions add pressure and energy terms that are analogous
to thermal motions if the turbulence has a Gaussian distribution.
Note
that the turbulent velocities may be organized, or lined up in certain
directions, if the gas is magnetically controlled.
This presents a
complication that is ignored.

\citet{HeilesTroland2005} discuss both turbulence and magnetic fields.
Their equation 34 gives the turbulent energy density in terms of the
one-dimensional turbulent velocity dispersion $\Delta V_{turb,1D}^2 $
(or standard deviation).
Note that we write velocities in terms of $u$
with the relationship $u^2  = 2\Delta V_{turb,1D}^2 $.
Their energy density can be rewritten as a pressure:
\begin{equation}
\begin{array}{ccl}
 P_{turb}  = \frac{F}{6}\rho \,u_{turb}^2&  =& 3.9 \times 10^{ - 10} F\left(
{\frac{{n_{tot} }}{{10^5 \;{\mathrm{cm}}^{ - 3} }}} \right)\left(
{\frac{{u_{turb} }}{{1\;{\mathrm{km}}\;{\mathrm{s}}^{ - 1} }}} \right)^2 \quad \left[
{{\mathrm{erg\; cm}}^{{\mathrm{ - 3}}} {\mathrm{;\ dyne\; cm}}^{ - 2}} \right] \\
&  =& 2.8 \times 10^6 \,\,F\,\left( {\frac{{n_{tot} }}{{10^5 \;{\mathrm{cm}}^{
- 3} }}} \right)\left( {\frac{{u_{turb} }}{{1\;{\mathrm{km}}\;{\mathrm{s}}^{ - 1}
}}} \right)^2 \quad \left[ {{\mathrm{cm}}^{ - 3} \;{\mathrm{K}}} \right] \\
 \end{array}
\end{equation}
where $n_{tot}$ is the total hydrogen density,
$u_{turb}$ is the turbulent velocity,
and He/H$ = 0.1$ was assumed.
The term $F$ accounts for how the turbulent
velocity field is ordered (\citealp{HeilesTroland2005}, their equation 34).
$F$ is 2 for turbulent velocities that are perpendicular to
the magnetic field
as in Alfven waves and $F$ is 3 for isotropic turbulent motions.
The default
value of $F = 3$ is changed by entering a new value as a second number on
this command line.

\subsection{Energy dissipation}

The \cdCommand{dissipate} option on the \cdCommand{turbulence}
command provides a way in include
conversion of wave energy into heat (see \citealp{BottorffFerland2002}).
When
the option is used a third number, the log of the scale length for the
dissipation in cm, must appear.
Then the turbulent velocity will have the form
\begin{equation}
u_{turb} (r) = u_{turb} (r_{\mathrm{o}} )\exp ( - \Delta r/r_{scale} )\quad
\mathrm{[cm\, s}^{-1}]% (55)
\end{equation}
where $u_{turb}(r_{\mathrm{o}})$ is the turbulence at the illuminated face
and $\Delta r$ is the depth into the cloud.
The wave mechanical energy is assumed to have been
converted into heat with a local heating rate given by
(\citealp{BottorffFerland2002})
\begin{equation}
G(r) = 3.45 \times 10^{ - 28} 2^{ - 3/2} u_{turb}^3 (r) \quad
\mathrm{[erg\, cm}^{-3} \mathrm{s}^{-1}]% (56)
\end{equation}

\subsection{Equipartition turbulent - magnetic pressures }

The \cdCommand{equipartition} option on the \cdCommand{turbulence} command sets the turbulent
velocity to an equipartition between magnetic and
turbulent energy densities.
That is,
\begin{equation}
\label{eqn:MagneticEquipartition}
P_{turb}  = \frac{F}{6}\rho \,u_{turb}^2  = P_{mag}  = \frac{{B^2 }}{{8\pi
}}\, \mathrm{[erg\, cm}^{-3}].% (54)
\end{equation}
The \cdCommand{magnetic field} command sets $B$.
The turbulent
velocity is determined from the magnetic field
assuming equation \ref{eqn:MagneticEquipartition}.
\citet{HeilesCrutcher2005} argue that the correlation between turbulent and
magnetic pressures, while true on average, does not hold in detail for
specific regions of the ISM. When this option is used, you can optionally add
the parameter $F$ on the command line (the default is $F=3$). The keyword \cdCommand{no pressure} is
supported, but not \cdCommand{dissipate}.

\subsection{Turbulence command heads up!}

\emph{N.B.!}  In the default (non-equipartition) form of the command, the turbulent velocity is the first number
on the command line.
The $F$ parameter is the second number.
The energy-dissipation scale length is the third number and must appear
after $u_{turb}$ and $F$.
These numbers can only be omitted from right to left.
In the equipartition case you can only specify $F$,
which then is the first number on the command line.
The energy-dissipation scale length is not supported in the equipartition case.
The keyword \cdCommand{vary} is only supported in the non-equipartition case,
in which case the turbulent velocity is varied.

The turbulent velocity must be less than the speed of light.
The code will stop if $u_{turb} \geq c$ is specified.

\section{vlaw alpha=-1}

This specifies a turbulent velocity that is a power law in radius.
The number is the power law $\alpha$ on radius.
It must be negative.
The turbulence will be given by
\begin{equation}
u = u_0 \left( {{\raise0.7ex\hbox{$r$} \!\mathord{\left/
 {\vphantom {r {r_0 }}}\right.\kern-\nulldelimiterspace}
\!\lower0.7ex\hbox{${r_0 }$}}} \right)^\alpha
\end{equation}
where $r_0$ in the inner radius and
$u_0$ is the initial turbulence specified with the
\cdCommand{turbulence} command.

\chapter{INCLUDING USER-DEFINED ATMOSPHERE GRIDS IN CLOUDY}
% !TEX root = hazy1.tex

\section{Introduction}

Cloudy now features the possibility of including user-defined stellar
atmosphere grids with an arbitrary number of dimensions (this is the number of
parameters that are varied in the grid). Using such a grid requires two steps.
The first is to create an ascii file with all the necessary information. The
format of this file will be described in Sect.~\ref{ascii}. The second step
is to compile this file into a binary atmosphere file, just like any other
atmosphere grid used by Cloudy. This is described in Sect.~\ref{compiling}.
The runtime behavior of the code is described in Sect.~\ref{runtime}.

\section{The ascii file}
\label{ascii}

Each stellar atmosphere grid needs to be defined in a single ascii file
containing all the necessary information. The format is illustrated in
Table~\ref{def}. The aim of this format is to keep the data in the ascii file
as close as possible to the original calculations. The file starts with some
general information, followed by a block of model parameters, the
frequency/wavelength grid, and finally each of the spectral energy
distributions (SEDs). No comments of any sort are allowed in the file. It is
implicitly assumed that each SED has been rebinned onto the same
frequency/wavelength grid. This grid does not need to coincide with the Cloudy
grid. The grid will be automatically rebinned onto the Cloudy grid during the
compilation stage described in Sect.~\ref{compiling}. The supplied
frequency/wavelength grid doesn't need to cover the entire Cloudy grid.
Extrapolation will be used to fill in the missing data. It is allowed for flux
points in the Wien tail to have zero flux. It is however the users
responsibility to assure that sufficient data are supplied to enable a safe
extrapolation of the spectra to the entire Cloudy grid.

I will now describe each of the items in the ascii file in more detail. The
first line contains the magic number that identifies the syntax version. This
number is the same in all ascii files that are used by Cloudy. It is checked
when Cloudy compiles the ascii file. This document describes version
\cdVariable{20060612}. The second line contains the number of parameters
\cdVariable{ndim} that are varied in the grid. Typically this number is 2, but
it may be any number $\geq$ 1 and $\leq$ \cdVariable{MDIM} (the latter
parameter is defined in \cdFilename{stars.h}). The default value for
\cdVariable{MDIM} is 4, which should be sufficient for all current grids. The
parameter can however be increased if the user wishes so. The code, as well as
the binary atmosphere files, will then have to be recompiled. The third lines
contains the number of parameters \cdVariable{npar} that are supplied for each
atmosphere model. This number may be larger than \cdVariable{ndim}, but not
smaller. The additional parameters are printed in the output, so using
\cdVariable{npar} $>$ \cdVariable{ndim} can be a useful way of supplying
additional information about the models. The number of parameters is limited
by the parameter \cdVariable{MDIM} defined in \cdFilename{stars.h}. Next
follow \cdVariable{npar} lines, each giving a label for that parameter. This
is used in the Cloudy output. The labels are completely free. However, if the
first label matches the string \cdVariable{Teff}, and the second label matches
\cdVariable{log(g)} (case sensitive), then special rules apply. We will refer
to this as a Teff-log(g) grid. See Sections~\ref{compiling} and \ref{runtime}
for further details. The next line gives the number of atmosphere models
\cdVariable{nmod} in the grid, followed on the next line by the number of
frequency/wavelength points \cdVariable{nfreq} in each of those models. In
order to keep the format of the ascii file as close as possible to the
original models, it is allowed to use either a frequency or wavelength grid in
arbitrary units. The same holds for the dependent variable, which may either
be a flux per frequency or wavelength unit. Both the dependent and independent
variable should be entered as linear quantities, so logarithmic units like
magnitudes are not supported. With this amount of freedom, the code needs to
know how to convert these variables to Cloudy's internal units. This is
defines on the next 4 lines. The next line should contain either
\cdVariable{nu} to state that the file contains a frequency grid, or
\cdVariable{lambda} to state that it contains a wavelength grid. The next line
contains a conversion factor to convert the numbers in the ascii file to a
frequency in Hz (case \cdVariable{nu}) or a wavelength in \AA\ (case
\cdVariable{lambda}). The conversion factor is multiplied with the numbers in
the ascii file. So, if the ascii file would contain a wavelength grid in
nanometer, then the conversion factor would have to be 10 to convert it into a
grid in \AA. The next line defines the data type of the dependent variable.
Legal values are \cdVariable{F\_nu, H\_nu} for fluxes per frequency unit, or
\cdVariable{F\_lambda, H\_lambda} for fluxes per wavelength unit. The next
line contains the conversion factor to convert the numbers in the ascii file
to a flux in erg\,cm$^{-2}$\,s$^{-1}$\,Hz$^{-1}$ (case \cdVariable{F\_nu,
  H\_nu}) or erg\,cm$^{-2}$\,s$^{-1}$\,\AA$^{-1}$ (case \cdVariable{F\_lambda,
  H\_lambda}). Again the conversion factors are multiplied with the numbers in
the ascii file. In the cases \cdVariable{H\_nu} and \cdVariable{H\_lambda} no
implicit factor $4\pi$ will be used! So if the ascii file contains Eddington
fluxes \cdVariable{H\_nu} in erg\,cm$^{-2}$\,s$^{-1}$\,\AA$^{-1}$\,sr$^{-1}$,
then the conversion factor should be 12.56637$\ldots$ All fluxes are
eventually renormalized in Cloudy, but nevertheless it is important to get
this conversion correct. The effective temperature of interpolated models is
internally checked by evaluating the integral:
\[ \int_0^\infty F_\nu d \nu = \sigma T_{\rm eff}^4. \]
If the result deviates too much from the expected value an error will be
printed. When compiling the grid, each model in the grid will be tested in the
same manner. A warning will be printed for each model that deviates from the
expected value. If all models generate such a warning, you should check your
normalization.

This concludes the general setup of the grid. The remainder of the file
contains the actual parameters and the data of the atmosphere models. These
numbers can be formatted in any way you like, as long as it complies with C
formatting rules. More in particular, you can supply more than 1 number per
line, as long as they are separated by whitespace. Several forms of Fortran
number formatting are not recognized in C! First, using ``D-format'' numbers
(e.g. 1.0d+20) is not allowed as C does not recognize this notation. Second,
Fortran may omit the ``e'' in numbers with large exponents (e.g. 1.0-102).
This notation is not recognized either by C. The ``e'' in the exponential
notation may be either lowercase or uppercase. In the ascii file, first the
model parameters must be supplied. This means that \cdVariable{nmod} $\times$
\cdVariable{npar} numbers must be entered, first all parameters for model \#1,
then the parameters for model \#2, etc. Next follows a block of
\cdVariable{nfreq} numbers defining the frequency/wavelength grid. Finally,
\cdVariable{nmod} blocks, each containing \cdVariable{nfreq} numbers, must be
entered which give the fluxes for each of the atmosphere models. This then
concludes the ascii file.

\begin{table}
  \caption[Example of an ascii file, derived from the kwerner.ascii file.]
  {Example of an ascii file, derived from the kwerner.ascii file.
    {\bf Comments are not allowed in the file and are only shown here for clarity.}
    \label{def}}
  \small
\begin{verbatim}
  20060612       # magic number
  2              # ndim
  2              # npar
  Teff           # label par 1
  log(g)         # label par 2
  20             # nmod
  513            # nfreq
  nu             # type of independent variable (nu or lambda)
  3.28984196e+15 # conversion factor for independent variable
  F_nu           # type of dependent variable (F_nu/H_nu or F_lambda/H_lambda)
  1.00000000e+00 # conversion factor for dependent variable
# nmod sets of npar model parameters
   80000. 5.0  80000. 6.0  80000. 7.0  80000. 8.0
  100000. 5.0 100000. 6.0 100000. 7.0 100000. 8.0
  120000. 6.0 120000. 7.0 120000. 8.0 140000. 6.0
  140000. 7.0 140000. 8.0 160000. 7.0 160000. 8.0
  180000. 7.0 180000. 8.0 200000. 7.0 200000. 8.0
# the frequency/wavelength grid, nfreq points
  1.0100000e-05  3.0396596e-04  6.0762797e-04  9.1129000e-04  9.8728144e-04
                   ...  <505 more frequency points>
  1.7655824e+02  1.7940598e+02  1.8225368e+02
# the SED for model 1, nfreq flux points
  6.2701018e-11  5.5260639e-08  2.1959737e-07  4.9197354e-07  5.7711065e-07
                      ...  <505 more flux points>
  9.0885815e-37  2.7039771e-37  7.9828291e-38
                 ... <the SED's for models 2 ... 19>
# the SED for model 20, nfreq flux points
  1.4960523e-10  1.3383193e-07  5.3344235e-07  1.1975753e-06  1.4049202e-06
                      ...  <505 more flux points>
  2.1177483e-13  1.3848120e-13  9.0415034e-14
\end{verbatim}
\end{table}

\section{Compiling the grid}
\label{compiling}

Ascii files containing the stellar atmosphere grid can have arbitrary names as
long as the filename extension is ``\cdFilename{.ascii}''. In the remainder we
will assume the file has been named \cdFilename{usergrid.ascii}. In order for
the grid to be used by Cloudy, it must be compiled into a binary format. This
format is platform dependent, so the binary file needs to be recreated for
every machine architecture that you wish to use it on. In order to compile the
file, go to the directory that contains the ascii file (this does not need to
be the Cloudy data directory), start up Cloudy and type the following command:

\begin{verbatim}
compile stars "usergrid.ascii"
\end{verbatim}

followed by two carriage returns. Cloudy will now create a file
\cdFilename{usergrid.mod}. If you did the above outside the normal Cloudy
directory, you now need to move the \cdFilename{usergrid.mod} file to the
Cloudy data directory. You can now use the grid by including the following in
your input file:

\begin{verbatim}
table star "usergrid.mod" 2.4e4 4.3
\end{verbatim}

In the case of a Teff-log(g) grid (see Section~\ref{ascii} for a definition)
you can supply only Teff, and log(g) will default to the highest value in the
grid. In all other cases you need to supply exactly \cdVariable{ndim}
parameters. See Section~\ref{sect:TableStars} for further details.

\section{Runtime behavior}
\label{runtime}

When Cloudy encounters the following command in its input:

\begin{verbatim}
table star "usergrid.mod" 24300 4.3
\end{verbatim}

it will try to interpolate the requested model using using as few models as
possible. This means that if an exact match could be found in the grid, only
that model will be used. If an exact match could be found for one parameter,
but not another, then only interpolation in the latter parameter will be
performed. In general no extrapolation will be performed. The only exception
is the log(g) parameter in Teff-log(g) grids for which more relaxed rules are
used. More on that later. First we will return to the example above. Let's
assume that the usergrid contains models for Teff = \{$\ldots$ 24000, 25000
$\ldots$\} and log(g) = \{$\ldots$ 4.0, 4.5 $\ldots$\}. Then Cloudy will
perform interpolation using the following 4 models: (Teff, log(g)) = (24000,
4.0), (24000, 4.5), (25000, 4.0), and (25000, 4.5). It will perform linear
interpolation on those models using the log of the flux. If any of the models
it needs for the interpolation is not present in the grid, the interpolation
will fail and Cloudy will stop. However, more relaxed rules exist for log(g)
in Teff-log(g) grids (see Section~\ref{ascii} for a definition). If it cannot
find a model with a given log(g), it will substitute the model with the
nearest log(g) instead. The reasoning behind this is that in general the
spectral energy distribution is not very sensitive to the value of log(g).
